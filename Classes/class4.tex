\documentclass[12pt,letterpaper]{article}
\usepackage{fullpage}
\usepackage[top=2cm, bottom=4.5cm, left=2.5cm, right=2.5cm]{geometry}
\usepackage{amsmath,amsthm,amsfonts,amssymb,amscd}
\usepackage{lastpage}
\usepackage{enumerate}
\usepackage{fancyhdr}
\usepackage{mathrsfs}
\usepackage{xcolor}
\usepackage{graphicx}
\usepackage{listings}
\usepackage{hyperref}

\hypersetup{%
  colorlinks=true,
  linkcolor=blue,
  linkbordercolor={0 0 1}
}
 
\renewcommand\lstlistingname{Algorithm}
\renewcommand\lstlistlistingname{Algorithms}
\def\lstlistingautorefname{Alg.}

\lstdefinestyle{Python}{
    language        = Python,
    frame           = lines, 
    basicstyle      = \footnotesize,
    keywordstyle    = \color{blue},
    stringstyle     = \color{green},
    commentstyle    = \color{red}\ttfamily
}

\setlength{\parindent}{0.0in}
\setlength{\parskip}{0.05in}

% Edit these as appropriate
\newcommand\course{Math 218A}
\newcommand\class{4}                  % <-- homework number


\pagestyle{fancyplain}
\headheight 35pt
\lhead{\NetIDa}
\lhead{\NetIDa\\\NetIDb}                 % <-- Comment this line out for problem sets (make sure you are person #1)
\chead{\textbf{\Large Class \class}}
\rhead{\course \\ \today}
\lfoot{}
\cfoot{}
\rfoot{\small\thepage}
\headsep 1.5em

\begin{document}

\section*{Review}
We know that top. spaces are: Hausdorff, 2nd countable, and locally euclidean.

Email him the homework. 

\subsection{Examples of top. mfds.}
\begin{itemize}
    \item countable set of pts on $\mathbf{R}^n$ (discrete pts & discrete topology) or any open subset of $\mathbf{R}^n$.
    \item Graphs. Let $ u\subset \mathbf{R}^n$ be an open set, then $f: u \rightarrow \mathbf{R}^m$ is a continuous map.
    \begin{equation}
        \Gamma(f) = \{ (x,y) | x \in U, y = f(x) \} \in \mathbf{R}^{n+m}
    \end{equation}
    with the subspace topology from $\mathbf{R}^{n+m}$. So, $\Gamma(f) $ is Hausdorff and 2nd countable. Also, locally euclidean. Locally euclidean bc has a global chart. The projection onto the first map is $\{ \phi, \Gamma(f), u \}$, where $\phi(x,y) = x$. So, $\phi$ is continuous with $\phi^{-1}(x) = (x, f(x))$ is continuous, so $\Gamma(f)$ is a top. mfd. of dim=n. Lagrangians are locally always graphs in euclidean!
    \item Spheres: The unit sphere (radius =1). For $\forall n \geq 0$, $\R \supset S^n = \{ (x_1, x_2, \dots , x_{n+1} | X_1^2 + \dots + x_{n+1}^2 \}$. Example of writing a space as a subspace of a higher dimension. Then, take the induced subspace topology of $\mathbb{R}^{n+1}$ . This is Hausdorff, 2nd countable. NOW, want to show locally euclidean, which is like taking the surface of the earth and mapping it into a plane. To do this...
    \begin{itemize}
        \item Cover $S^n$ by the union of 2 open subsets. One is:
        \begin{equation}
            u_+ = S^n \backslash \{ 0,0, \dots, 0, -1\}
        \end{equation}
        \begin{equation}
            u_- = \backslash \{ 0,0, \dots, 0, +1\}
        \end{equation}
        See the image from class for a pictorial representation of this. 
        \item Can define the sphere in $\mathbb{R}^{n+1}$ coordinates, and want to map into $\mathbb{R}^n$ by $\phi_+(x_1, \dots, x_{n+1}) = \frac{1}{1 + x_{n+1}} \underbrace{(x_1, \dots, x_n)}_{y_1, \dots, y_n}$. Again, see the class notes for a diagram of what he meant by this. This works with everything except for the south pole (where the projection is coming from, and the image is depicting $\phi_-$, but I have from the south pole here). 
        \item So, $\phi_{\pm}$ are continuous & invertible.  So, $\phi_{\pm}(y_1, \dots, y_n) = \frac{1}{1 +|y|^2}(2y_1, 2y_2, \dots, 2y_n)$ where $|y|^2 = y_1^2 + \dots + y_n^2$.
    \end{itemize}
    \item Projection spaces: The n-dim real projection space $\matbb{R} \mathbb{P}^n$ by definition is the 1-dim linear subspaces (i.e. lines) through the origin in $\R^{n+1}$ endowed with the quotient topology. 
    \begin{equation}
        \mathbb{R} \mathbb{P}^n := \R^{n+1} \backslash \{ 0\}
    \end{equation}
    Essentially looking for the set of lines in space. AND using the equivalence relation: $(x_1, \dots, x_{n+1} \sim (tx_1, \dots, tx_{n+1}) \forall t \neq 0$. So,
    \begin{equation}
        \mathbb{R} \mathbb{P}^1 = \mathbb{R}^2 \backslash \{ 0 \} 
    \end{equation}
    More generally, $\mathbb{R}\mathbb{P}^n = \frac{S^n}{x \sim -x}$. Not a very big deal and has a famous cousin, $\mathbb{C}\mathbb{P}^n$ that is an n-dim complex mfd. projective space by letting $x_i \in \mathbb{C}$. There's also quarternionic projective space, but we're keeping it chill. However, we want euclidean charts to describe subsets of these spaces. 
    
    We need to show locally euclidea before we define the open sets. Can define an equivalence class, which is like defining an entire line as a point. Let
    $[x_1: x_2: \dots : x_{n+1}] $ (using brackets to describe an equivalence class). Denote the element in $\mathbb{R}\mathbb{P}^n$ containing the pt. $(x_1, \dots, x_{n+1})$ in $\mathbb{R}^{n+1} \backslash \{0\}$.
    
    Now, define the open sets, $u_i = \{ [x_1:, \dots: x_{n+1}]| x_i \neq 0\}$ for $i=1, \dots, n+1$. 
    
    The charts $\{ \phi_i, u_i, \mathbb{R}^n\}$ are given. So, 
    \begin{equation}
        \phi_i([x_1: \dots : x_{n+1}]) = (\frac{x_1}{x_i}, \dots, \frac{x_{i-1}}{x_i}, \frac{x_{i+1}}{x_i}, \dots, \frac{x_{n+1}}{x_i})
    \end{equation}
    And its inverse:
    \begin{equation}
        \phi_i^{-1}(y_1, \dots, y_n) = [y_1: \dots : y_{i -1}: 1 : y_{i+1}: \dots : y_n}]
    \end{equation}
    This is a systematic method of constructing mfds. 
    
    \item Product Mfds: If $M_1$ and $M_2$ are top. mfds. of dim $n_1$ and $n_2$, then product $M-1 x M_2$ endowed w/ product topology is a top mfd. of dim $n_1 + n_2$. An example is a n-dim torus, $T^n = S^1 \times S^2 \times \dots S^n$
    
    \item Open subsets of a top mfd. Ex. The general linear group $GL(n, \mathbb{R})$. So, let $M(n;\mathbb{R}) = $ set of all n x n real matrices. This is in contrast to the isomorphic $\mathbb{R}^{n^2}$ group, $GL(n;\mathbb{R}) = \{ A \leftarrow M(n, \mathbb{R}) | \det A \neq 0\}$
    
    
\end{itemize}


\end{document}
