\documentclass[12pt,letterpaper]{article}
\usepackage{fullpage}
\usepackage[top=2cm, bottom=4.5cm, left=2.5cm, right=2.5cm]{geometry}
\usepackage{amsmath,amsthm,amsfonts,amssymb,amscd}
\usepackage{lastpage}
\usepackage{enumerate}
\usepackage{fancyhdr}
\usepackage{mathrsfs}
\usepackage{xcolor}
\usepackage{graphicx}
\usepackage{listings}
\usepackage{hyperref}

\hypersetup{%
  colorlinks=true,
  linkcolor=blue,
  linkbordercolor={0 0 1}
}
 
\renewcommand\lstlistingname{Algorithm}
\renewcommand\lstlistlistingname{Algorithms}
\def\lstlistingautorefname{Alg.}

\lstdefinestyle{Python}{
    language        = Python,
    frame           = lines, 
    basicstyle      = \footnotesize,
    keywordstyle    = \color{blue},
    stringstyle     = \color{green},
    commentstyle    = \color{red}\ttfamily
}

\setlength{\parindent}{0.0in}
\setlength{\parskip}{0.05in}

% Edit these as appropriate
\newcommand\course{Math 218A}
\newcommand\class{20}                  % <-- homework number


\pagestyle{fancyplain}
\headheight 35pt
\lhead{\NetIDa}
\lhead{\NetIDa\\\NetIDb}                 % <-- Comment this line out for problem sets (make sure you are person #1)
\chead{\textbf{\Large Class \class}}
\rhead{\course \\ \today}
\lfoot{}
\cfoot{}
\rfoot{\small\thepage}
\headsep 1.5em

\begin{document}

\section*{Smooth Submersions}

\textbf{Defn.:} $F: M \rightarrow N$ is a smooth submersion if $dF_P$ is surjective $\forall p \in M$. (In this case, $rank F := rank dF_P = \dim N$.)

Property of smooth submersions: They have \textit{local sections}. What does \textit{section} mean, though? \textbf{Defn.:} For $\pi : M \rightarrow N$, a continuous map, a section of $\pi$ is a continuous right inverse for $\pi$, i.e. a continuous map $\sigma: N \rightarrow M$ s.t. $\pi \circ \sigma = Id_N$. 

\textbf{Defn: Local section.} A local section of $\pi$ is a continuous map $\sigma: U \mapsto M$ defined on an open nbhrd. $U \subseteq M$ satisfying $\pi \circ \sigma = Id_U$. There's an alternative way to define a smooth submersion using local sections.

\textbf{Thrm. Local Section Thrm.} Suppose $M, N$ are smooth mfds. and $\pi: M \rightarrow N$ is a smooth map. Then $\pi$ is a smooth submersion s.t. every pt. of $M$ is a in the image of a smooth local section of $\pi$. A smooth submersion is equivalent to this.

\textbf{Proof:} (forward first) Suppose $\pi$ is a smooth submersion, we want to construct a local section, so let $p \in M$ and subbmersion $q = \pi(p) \in N$. By the canonical submersion theorem, we can choose coordinates locally s.t. $(x_1, \dots, x_m)$ centered at $p$, we choose $(y_1, \dots, y_n)$ centered at $q$, s.t. $\pi(x_1, \dots, x_m) = (x_1, \dots, x_n)$ as our "canonical projection". Now define an $\epsilon$-nbhd., $U_{\epsilon} = \{ x | |x_i| < \epsilon$ for $i = 1, \dots, m \}$ whose image (under $\pi$?) is $V_{\epsilon} = \{ y | |y_i| < \epsilon$ for $i = 1, \dots, n\}$. THEN the map $\sigma: V_{\epsilon} \rightarrow U_{\epsilon}$ given by $\sigma(y_1,  \dots, y_n) = (y_1, \dots, y_n, 0 , \dots, 0)$ is a smooth local section of $\pi$ satisfying $\sigma(q) = p$. 

(backwards direction) Assume each pt. in $M$ is in the image of a smooth local section. Given $p \in M$, let $\sigma: U \rightarrow M$ be a smooth local section s.t. $\sigma (q) = p$, i.e. $q = \pi(\sigma(q)) = \pi(p) \in N$, bc $\pi \circ \sigma$ is the identity. We have $\pi \circ \sigma = Id_U$ as the lcoal section. So, $d\pi_p \circ d\sigma_q = Id_{T_qN}$ this is the identity and $d\pi_p$ is surjective $\forall p \in M$. 

A more relaxed version of smooth submersion is a topological submersion, which uses a local section that is just continuous (not necessarily smooth). 

So, this theorem motivates this definition for a topological submersion. \textbf{Defn:Topological submersion}  is a continuous map $\pi: M \rightarrow N$ is a top. submersion if every pt. in $M$ is in the image of a continuous local section. Want to consider inverse map that's continuous but not smooth. 

\textit{Example.} $\pi: \mathbb{R}^2 \rightarrow \mathbb{R}$ and $(x_1, x_2) \mapsto x_1^3$, then $\pi$ is not a smooth submersion (at the zero!). If the lcoal sections are $\sigma: \mathbb{R} \rightarrow \mathbb{R}^2$ and $y \mapsto \sigma_{x_2}(y) = (y^{1/3}, x_2)$, where $x_2$ is a continuous map and parameter. This is a topological submersion that's not a smooth submersion. 

\textit{Interesting statement:} (Lee Prop. 4.28) If $\pi : M \rightarrow N$ is a smooth submersion, and also subjective, then $\pi$ is a quotient map, i.e. $N$ has the quotient top. determined by $\pi$. Like the case of an embedding where image corresponds to subspace top. 

\textbf{Level Sets.} Submfds. are often defined as the solution sets of equations. e.g. $S^n \subseteq \mathbb{R}^{n+1}$, $f: \mathbb{R}^{n+1} \rightarrow \mathbb{R}$, and $x=(x_1, \dots, x_{n+1}) \mapsto f(x) = |x|^2$. The unit sphere is defined to $f(x)=1.$ Thus, $S^n$ is an embedded submfd of $\mathbb{R}^{n+1}$ also in the \textit{level set} of a function $f^{-1}(1)$. 

\textbf{Defn.:} For any map $F: M \rightarrow N$ and $q \in N$, we say $F^{-1}(q) = \{ p \in M | F(p) = q\}$ is a level set of $F$. See class picture. 
\begin{itemize}
    \item If $N = \mathbb{R}^k$ and $q = 0$, the level set of $F^{-1}(0)$ is typically called the zero set of $F$.
    \item Not all level sets of smooth functions are smooth submfds...
\end{itemize}

Example. $F(x,y) = x^2 - y^2$, zero set of $F$ is $F^{-1}(0)$ corresponds to $x^2 = y^2$ so $x=\pm y$.

Now, by partition of unity argument (Lee Thrm. 2.29), every closed subset of $M$ can be expressed as the zero set of some smooth (non-negative) real-valued function. 

Finally, \textbf{Thrm. (Lee 5.12) Constant Rank Level Set Thrm}: Let $M$, $N$ smooth mfds. and $F: M \rightarrow N$ smooth map with constant $rank=k$. Then each level set of $F$ is a properly embedded submfd. Constant rank corresponds to $rank dF_p$ with codimension $k$ in $M$. Imposing conditions to be able to shrink the dimensionality.

\end{document}
