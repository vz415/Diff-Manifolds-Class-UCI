\documentclass[12pt,letterpaper]{article}
\usepackage{fullpage}
\usepackage[top=2cm, bottom=4.5cm, left=2.5cm, right=2.5cm]{geometry}
\usepackage{amsmath,amsthm,amsfonts,amssymb,amscd}
\usepackage{lastpage}
\usepackage{enumerate}
\usepackage{fancyhdr}
\usepackage{mathrsfs}
\usepackage{xcolor}
\usepackage{graphicx}
\usepackage{listings}
\usepackage{hyperref}

\hypersetup{%
  colorlinks=true,
  linkcolor=blue,
  linkbordercolor={0 0 1}
}
 
\renewcommand\lstlistingname{Algorithm}
\renewcommand\lstlistlistingname{Algorithms}
\def\lstlistingautorefname{Alg.}

\lstdefinestyle{Python}{
    language        = Python,
    frame           = lines, 
    basicstyle      = \footnotesize,
    keywordstyle    = \color{blue},
    stringstyle     = \color{green},
    commentstyle    = \color{red}\ttfamily
}

\setlength{\parindent}{0.0in}
\setlength{\parskip}{0.05in}

% Edit these as appropriate
\newcommand\course{Math 218A}
\newcommand\class{17}                  % <-- homework number

\pagestyle{fancyplain}
\headheight 35pt
\lhead{\NetIDa}
\lhead{\NetIDa\\\NetIDb}                 % <-- Comment this line out for problem sets (make sure you are person #1)
\chead{\textbf{\Large Class \class}}
\rhead{\course \\ \today}
\lfoot{}
\cfoot{}
\rfoot{\small\thepage}
\headsep 1.5em

\begin{document}

\section*{Embedded Submanifold}

Def: Let $M$ be a smooth mfd. of dim n. For $k < n$, a subset $S \subset M$ is a k-dim embedded submanifold of $M$ if $S$ satisfies the slice chart condition. i.e. $\forall p \in S, \exists $ a chart $\{\phi, U, V\}$ around $p \in M$ s.t. $\phi (U \cap S) =  V \cap \{ \mathbb{R}^K \times \{0\}\}= \{ x \in \phi(u) | x_{k+1} = \dots = x_n = 0\}$.

Remarks
\begin{itemize}
    \item Lee's defn. of an embedded submfd. is a subset $S \subseteq M$ that is a mfd. in the subspace topology endowed with a smooth structure w.r.t which map $i: S \mapsto M$ is a smooth embedding. 
    \item Two defns are equivalent! We will show embedded submfds. (slice chart) is the same as Lee's defn. of smooh embeddings
    \item Embedded submfds are also called "regular submanifolds".
    \item For $S \subset M$ an embedded submfd of $dim = k$ IF $\dim M = n$, then the difference $(n-k)$ is called the \textit{codimension} of $S$. 
    \item An embeded submfd. of codimension=1 is called an embedded hypersurface.
    \item The mfd. $M$ containing $S$ is sometimes called "the ambient" mfd. of $S$. 
\end{itemize}

\textit{Theorem for embedded submfd. with slice chart:} Let $S$ be a k-dim embedded submfd. of $M$, then with the subspace topology $S$ admits a smooth structure s.t.:
\begin{enumerate}
    \item $S$ is a smooth mfd. of $\dim = k$
    \item The inclusion map, $i: S \mapsto M$ is a smooth embedding.
\end{enumerate}

Comments: The Defn. of embedded submfd. (slice chart) $S \implies S$ has slice charts induced from $M$. So, (1)$\implies S$ is a smooth mfd. fro these charts. AND with this smooth structure, we can consider the smoothness of the map $i: S \mapsto M$. AND by (2), $i$ is a smooth embedding. Also, it turns out that there is a unique smooth structure on $S$ s.t. $i: S \mapsto M$ is a smooth embedding. Up to diffeomorphism, this smooth structure is the slice chart from the definition. 

\textbf{Proof of (1)}: 
$S$ has to satisfy all smooth mfd. axioms (Hausdorf, 2nd countable) inherited from subspace topology. To construct local charts, let $\pi: \mathbb{R}^n \rightarrow \mathbb{R}^k$ and $(x_1, \dots, x_n) \mapsto (x_1, \dots, x_k)$ and $i: \mathbb{R}^k \rightarrow \mathbb{R}^n$ and $(x_1, \dots, x_k) \mapsto (x_1, \dots, x_k, 0, \dots, 0)$. 

For any chart $(\phi, U, V)$ of $M$ satisfying the slice chart condition. Now, let $X = U \cap S$ and $\psi = \pi \circ \phi$ and $Y = \pi \circ \phi(x)$. Finally, the inclusion is $\psi^{-1} = \phi^{-1} \circ i$ and this all implies that $(\psi, X, Y)$ is a chart on $S$. 
\begin{itemize}
    \item Moreover, charts of this type are compatible, since...
    \begin{align*}
        \psi_{\alpha \beta} &= \psi_{\beta} \circ \psi_{\alpha}^{-1} \\
        &= (\pi \circ \phi_{\beta}) \circ (\phi_{\alpha}^{-1} \circ i) \\
        &= \pi \circ \phi_{\alpha \beta} \circ i \implies \text{is a smooth map from } Y_{\alpha} \text{to } Y_{\beta}
    \end{align*}
\end{itemize}

\textbf{Proof of (2)}: 
Check $i: S \mapsto M$ is a smooth embedding. With $i(s)$ endowed with subspace topology, the map $i: S \mapsto i(S) \subset M \forall p \in S\subset M$ and $i(p) = p$ is treatedd as a pt. in $M$. And the map is a homeo.

It is also an immersion since for each pair of charts. So what is $\hat{i}$? Well, $\hat{i} =\phi \circ i \circ \psi^{-1}$ where $\psi^{-1} =  \phi^{-1} \circ i$ and $\hat{i}$ is the canoncical immersion map and its differential, $d\hat{i}$ is injective. 

\textbf{Proposition:} The slice chart is the unique smooth structure that gives a smooth embedding. 

\textbf{Proof:}Suppose there is some other topology & smooth structure on $S$ s.t. $\Tilde{i} : \Tilde{S}\rightarrow M$ is also a smooth embedding. Here, $\Tilde{S}$ denotes the same set as $S$, but with a different smooth structure. This inclusion map, $\Tilde{i}(\Tilde{S}) = S$, where $S$ is the image on $M$. Can also consider $\Tilde{i}$ as the smooth map from $\Tilde{S} \mapsto S$.

Therefore, for $\forall p \in S$, we can represent $d\Tilde{i}_P : T_P\Tilde{S} \rightarrow T_P M$ as a composition $T_P \Tilde{S} \xrightarrow{d\Tilde{i}_P} T_P S \xrightarrow{di_p} T_P M$ since $d\Tilde{i}_P: T_P \Tilde{S} \rightarrow T_P M$  is injective, this implies $d\hat{i}_p :T_P \Tilde{S} \rightarrow T_P S$ is injective. So this all implies $\Tilde{i}: \Tilde{S} \rightarrow S$ is an immersion where $\Tilde{i}$ is a local diffeo. But $\Tilde{i}(\Tilde{S}) = S$ is bijective. Thus, $\Tilde{i}$ is a diffeo.

\end{document}
