\documentclass[12pt,letterpaper]{article}
\usepackage{fullpage}
\usepackage[top=2cm, bottom=4.5cm, left=2.5cm, right=2.5cm]{geometry}
\usepackage{amsmath,amsthm,amsfonts,amssymb,amscd}
\usepackage{lastpage}
\usepackage{enumerate}
\usepackage{fancyhdr}
\usepackage{mathrsfs}
\usepackage{xcolor}
\usepackage{graphicx}
\usepackage{listings}
\usepackage{hyperref}

\hypersetup{%
  colorlinks=true,
  linkcolor=blue,
  linkbordercolor={0 0 1}
}
 
\renewcommand\lstlistingname{Algorithm}
\renewcommand\lstlistlistingname{Algorithms}
\def\lstlistingautorefname{Alg.}

\lstdefinestyle{Python}{
    language        = Python,
    frame           = lines, 
    basicstyle      = \footnotesize,
    keywordstyle    = \color{blue},
    stringstyle     = \color{green},
    commentstyle    = \color{red}\ttfamily
}

\setlength{\parindent}{0.0in}
\setlength{\parskip}{0.05in}

% Edit these as appropriate
\newcommand\course{Math 218A}
\newcommand\class{21}                  % <-- homework number


\pagestyle{fancyplain}
\headheight 35pt
\lhead{\NetIDa}
\lhead{\NetIDa\\\NetIDb}                 % <-- Comment this line out for problem sets (make sure you are person #1)
\chead{\textbf{\Large Class \class}}
\rhead{\course \\ \today}
\lfoot{}
\cfoot{}
\rfoot{\small\thepage}
\headsep 1.5em

\begin{document}

\section*{Review: Smooth Maps and regular/critical values}
For smooth map $F: M \rightarrow N$, 
\begin{itemize}
    \item $q \in N$ is called a regular value of $F$ if $F$ is a submersion (differential) $\forall p \in F^{-1}(q)$. 
    \item $q \in N$ is called a critical value of $F$ if $q$ is not a regular value...
\end{itemize}
Note: this is different than critical \textunderscore{points}!
\begin{itemize}
    \item $p \in M$ is called a regular point of $F$ if $dF_P$ is surjective.
    \item $p \in M$ is called a critical point of $F$ if it is not a regular point.
\end{itemize}

\textbf{Thrm.:} If $q \in N$ is a regular value of $F: M \rightarrow N$, then $S=F^{-1}(q)$ is an embedded submfd. of $M$. 

Question: Given a smooth map $F: M\rightarrow N$, does $F$ have regular values? Answer: By Sard's thrm. (1942), roughly, almost all points in $F(M)$ are regular values. i.e. the size (set) of the critical value is "very small" in $N$. What is "very small"?

Need a kind of "measure" to tell us the size of the set. However, we have yet to develop a way to measure volume on a smooth mfd. qualitatively. We haven't given our manifolds things like iner products... (this is a subject of riemannian geometry). 

On $\mathbb{R}^n$, we have a "measure zero" set. \textbf{Defn.:} Given a set $A \subset \mathbb{R}^n$ is called a "measure zero" if $\forall \epsilon > 0, \exists $ a countable family of open sets $V_j \subset \mathbb{R}^n$ s.t.:
\begin{enumerate}
    \item $A \subset \underset{j}{U} V_j$
    \item $\sum_j Vol(v_j) < \epsilon $ 
\end{enumerate}

Example of measure zero sets. On $\mathbb{R}^1$, (i) have a finite set of pts. On $\mathbb{Q}$, (iii) \textit{cantor set}, which if you take the closed interval from zero to one, then middle third interval, then repeat, and keep going for a while and then take the intersection. It's an uncountable set and is \textit{nowhere dense}.

On $\mathbb{R}^n$, any k-dim ($k<n$), subspace of $\mathbb{R}^n$ has a measure zero. (i.e. lines have no area, planes have no volume).

\textbf{Two properties of measure zero sets }.
\begin{enumerate}
    \item A countable union of measure zero sets is still measure zero.
    \item (Lee prop. 6.5) If $A \subset \mathbb{R}^n$ is a measure zero set and $F: A \rightarrow \mathbb{R}^n$ is a smooth map, then $F(A)$ has measure zero. 
\end{enumerate}

We'll extend the notion of "measure zero" to subsets of mfds. Recall: Any mfd. $M$ admits an atlas of countable charts and for each chart we can identify an open set in $M$ with an open set in $\mathbb{R}^n$. 

Now, \textbf{Defn.:} Let $M$ be a smooth mfd. of $\dim m = n$, we say a subset $A \subseteq M$ has measure zero in $M$ if for any smooth chart $(\phi, U, V)$ on $M$, the subset $\phi (U \cap A) \subseteq \mathbb{R}^n$ has a n-dim. measure zero.

Remark: The notion of "measure zero" on mfd. is a diffeomorphism invariant. (see picture)

So, $\phi_{\alpha \beta} = \phi_{\beta} \phi_{\alpha}^{-1}$ is a smooth map. By (2) measure zero on $V_{\alpha}$ is a masure zero on $V_{\beta}$. 

Suppose $F: M\rightarrowN$ is a smooth map where $\dim M = \dim N = n$. If $A \subseteq M$ is a measure zero, then $F(A)$ has measure zero in $N$. 

Proof: Let $\{\phi_i, U_i, V_i\}$ be a countable cover of $N$, we need to show for each $(\phi_j, X_j, Y_j)$ on $N$, the set $\psi_j (F(A) \cap X_j)$ has measure zero in $\mathbb{R}^n$. We can express $\psi_j$ as a countable union of subsets. So, $\psi_j \circ F \circ \phi_i^{-1}(\phi_i (A \cap U_i \cap F^{-1}(X_j))$, each of the charts has measure zero.

\textbf{Thrm: Sard's Thrm.} If $F: M \rightarrow N$ is a smooth map and $C$ is the set of all critical pts. of $F$ in $M$, then $F(C)$ is a measure zero set in $N$. 

Remark: Thrm does not claim the set of critical pts. is measure zero... Ex. A constant map where $F(p) = q_0 \in N \forall p \in M$, so all $p \in M$ are critical points. 

However, $F(M)$ still has measure zero. 

\textbf{Corollary to Sard's Thrm.} If $F: M \rightarrow N$ is smooth and $\dim M < \dim N$, then $F(M)$ has measure zero in $N$. \textbf{Proof}: 
\begin{enumerate}
    \item Show any $p \in M$ is a critical pt. Proof: by contradiction, suppose $\exists p \in M$ a regular pt., then $dF_p$ is surjective (impossible since linear map!). Not possible since $\dim M < \dim N$. 
    \item Sard's thrm. says $F(C) = F(M)$ is a measure zero set. 
\end{enumerate}
Proof of Sard's Thrm in Lee Thrm. 6.10. It's a proof by induction that, let $m = \dim M$ and $n = \dim N$, then (1) $m=0$ for $n=0$, $F$ has no critical pts. If $n>0$ image of $F$ has measure zero. (2) For $m>1$ suppose hodors for domain of $\dim < m$, then show holds for domain of $\dim m =m$.

\end{document}
