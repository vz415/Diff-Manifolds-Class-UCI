\documentclass[12pt,letterpaper]{article}
\usepackage{fullpage}
\usepackage[top=2cm, bottom=4.5cm, left=2.5cm, right=2.5cm]{geometry}
\usepackage{amsmath,amsthm,amsfonts,amssymb,amscd}
\usepackage{lastpage}
\usepackage{enumerate}
\usepackage{fancyhdr}
\usepackage{mathrsfs}
\usepackage{xcolor}
\usepackage{graphicx}
\usepackage{listings}
\usepackage{hyperref}

\hypersetup{%
  colorlinks=true,
  linkcolor=blue,
  linkbordercolor={0 0 1}
}
 
\renewcommand\lstlistingname{Algorithm}
\renewcommand\lstlistlistingname{Algorithms}
\def\lstlistingautorefname{Alg.}

\lstdefinestyle{Python}{
    language        = Python,
    frame           = lines, 
    basicstyle      = \footnotesize,
    keywordstyle    = \color{blue},
    stringstyle     = \color{green},
    commentstyle    = \color{red}\ttfamily
}

\setlength{\parindent}{0.0in}
\setlength{\parskip}{0.05in}

% Edit these as appropriate
\newcommand\course{Math 218A}
\newcommand\class{2}                  % <-- homework number


\pagestyle{fancyplain}
\headheight 35pt
\lhead{\NetIDa}
\lhead{\NetIDa\\\NetIDb}                 % <-- Comment this line out for problem sets (make sure you are person #1)
\chead{\textbf{\Large Class \class}}
\rhead{\course \\ \today}
\lfoot{}
\cfoot{}
\rfoot{\small\thepage}
\headsep 1.5em

\begin{document}

\section*{Review continued}

Smoothness of a Euclidean Space.

The differential of f

The directional derivative vectorr of f at pt. x along the direction of v.

Chain Rule

If $f: u \rightarrow v$ and $g: v \rightarrow w$

Diffesomorphisms (make sure not to confuse with homeomorphism)

Def: A smooth map $f: u \rightarrow v$ is a diffeomorphism if f is:

1. bijection (1-1 AKA onto, onto)

2. $f^{-1} : v \rightarrow u$ is also smooth.

Clearly, if $f: u \rightarrow v$ is a diffeomorphism, then... AND if, $f: u \rightarrow v$, $g: v \rightarrow w$ are diffeos, then $g \circ f$ is a diffeomorphism. AND if $f: u \rightarrow v$ is a diffeo, then, $\forall x \in U$, the linear map (invertible matrix), $df_x$ is an isomorphism. then $dim(U) = dim(V)$

PROOF: Follow, $f^{-1} \circ f = \mathbf{I}d_u$, and $f \circ f^{-1} = \mathbf{I}d_v$, so the identity matrix wil, $df^{-1}_{f(x)} \circ df_x = I d_{\mathbf{R^n}}$ and $df^{-1}_{f(x)} \circ df_x = I d_{\mathbf{R^m}}$

edit this later ^.

An isomorphism is not a diffeomorphism. 

Ex. The inverse statement is not true! 

\begin{equation}
    f: \mathbf{R}^2 \backslash \{0\} \rightarrow \mathbf{R}^2 \backslash \{0\} \\
    (x_1, x_2) \rightarrow (x_1^2 -x_2^2, 2x_1x_2) \\
    \forall x \in \mathbf{R}^2 \backslash \{0\}
    
\end{equation}
Soo, $df_x$ is an isomorphism. However, $f(x) = f(-x)$.

Comment: diffeo. btwn different sets, $u$, $v$ in $\mathbf{R}^n$ gives a notion of when two open sets "look the same" gives an equivalence relation in smooth category. 

\subsection{More General Geometrical Objects and Maps}
Maps between open sets between geometrical objects. Generalize the notion of open maps between open sets (broader than just $\mathbf{R}^n$, but more generally on geometrical objects. 

Reverse question: What type of geometrical spaces can we define smooth maps, or, the relaxed condition, continuous maps? (Ohh didn't know continuous maps were a relaxation of smooth maps!)

Topology: Create an open space to define maps on open spaces. Toplogical spaces!! Allows us to define continuous maps. 

Toplogical manifolds: subset of topological spaces with additional conditions. We need to define smooth maps on the spaces we are studying. 

Smooth manifolds: Subset of top. manifolds, allows to define smoothness of maps)

\section{Review of Topology}

Continuous functions. In calculus we use the limit definition for this, but can also say: $f: \mathbf{R} \rightarrw \mathbf{R}$ is continuous if for any open set $v \subset \mathbf{R}$, the preimage of $f^{-1}(v)$ is also open. 

Def: A topological space $X$ is a set $X$ together with a collection of subsets of $X$ (called the topology of $X$) whose elements are called open sets, s.t., 1. the empty set ${0}$ and whole set, $X \in A$, 2. any union (might be infinite) of open sets is also open. 3. Any finite intersection of open sets is also an open set. 

toplogy is all about open sets. If you have a topology, you have an open set. Remark: The complement of an open set is a closed set. 

Ex: Special topology. 1. $A = ${all subsets of $X$} is call ed the "Discrete Topolgy" - every subset is both open and closed.  2. "Trivial toplogy" this corresponds to $A = {\zero , X}$. This is the trivial. 3. Metric topology of $\mathbf{R}^n$ is the open sets defined by open balls around any point.  4. Induced topology: case where , Let $Y \subset X$, a subset in $Y$ is open if it is of the form $U \cap Y$ for some open st $U$ in $X$.


With topology, we have open sets. Def: A map $f: X \rightarrow Y$ between topological spaces is continuous if for any open set $v \subset Y$ the preimage $f^{-1}(v) $ is open in $X$. Notion of equivalence for topological objects.

Qualification. Def: A continous map $f: X \rightarrow Y$ between two topological spaces is called a homeomorphism if 1. f is bijective, 2. $f^{-1}$ is continuous. 

Remark: for $f$ a diffeo, $f: U \in \mathbf{R}^n \rightarrow V \rightarrow \mathbf{R}^m$, then n=m, and the proof is a lot harder!

For homeomorphism, we have the same theorem with diffeo replaced by homeo. 

Theorem is called the "Invariance of Domain". Can look this up but they talk about it in algebraic topology. 

Define connectedness, compactness, etc. All are in the appendix of Lee's book. 

\end{document}
