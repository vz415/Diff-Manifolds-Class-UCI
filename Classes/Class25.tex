\documentclass[12pt,letterpaper]{article}
\usepackage{fullpage}
\usepackage[top=2cm, bottom=4.5cm, left=2.5cm, right=2.5cm]{geometry}
\usepackage{amsmath,amsthm,amsfonts,amssymb,amscd}
\usepackage{lastpage}
\usepackage{enumerate}
\usepackage{fancyhdr}
\usepackage{mathrsfs}
\usepackage{xcolor}
\usepackage{graphicx}
\usepackage{listings}
\usepackage{hyperref}
\usepackage{stackengine}
\newcommand\dhookrightarrow{\mathrel{%
  \ensurestackMath{\stackanchor[.1ex]{\hookrightarrow}{\hookrightarrow}}
}}

\hypersetup{%
  colorlinks=true,
  linkcolor=blue,
  linkbordercolor={0 0 1}
}
 
\renewcommand\lstlistingname{Algorithm}
\renewcommand\lstlistlistingname{Algorithms}
\def\lstlistingautorefname{Alg.}

\lstdefinestyle{Python}{
    language        = Python,
    frame           = lines, 
    basicstyle      = \footnotesize,
    keywordstyle    = \color{blue},
    stringstyle     = \color{green},
    commentstyle    = \color{red}\ttfamily
}

\setlength{\parindent}{0.0in}
\setlength{\parskip}{0.05in}

% Edit these as appropriate
\newcommand\course{Math 218A}
\newcommand\class{25}                  % <-- homework number


\pagestyle{fancyplain}
\headheight 35pt
\lhead{\NetIDa}
\lhead{\NetIDa\\\NetIDb}                 % <-- Comment this line out for problem sets (make sure you are person #1)
\chead{\textbf{\Large Class \class}}
\rhead{\course \\ \today}
\lfoot{}
\cfoot{}
\rfoot{\small\thepage}
\headsep 1.5em

\begin{document}

HW4: Due last day of class (12/3). Lee Chapter 6:2,9,10,13(a,b),14

Late HW no later than 12/10.

\section*{Review: Sard's Theorem}

It's the set of critical values is the set of measure zero. We used this to show Whitney's embedding Thrm. Should know that every smooth compact n-mfd. has a smooth embedding into $\mathbb{R}^{2n+1}$  or immersion into $\mathbb{R}^{2n}$. This can be improved (but we didn't prove) that this can be improved to embdding into $\mathbb{R}^{2n}$ (Immersion into $\mathbb{R}^{2n-1}$ and assume $n>1$). In the prrof, we used that when $M$ is compact, embedding map $F$ is (1) injective and (2) immersion. These allow us to conclude that $F$ is an embedding. 

However, when $M$ is not compact, have to add an extra condition and need to show that $F$ is a proper map (case when given any compact set, the preimae of the compact set is also a compact set - see Lee Lemma 6.14). 

\section{Another Application of Sard's Thrm.}

Concerns intersection of two spaces.
\begin{enumerate}
    \item Submfd. $S \subseteq M$ and then $S \cap M = S$ is an intersection.
    \item $s, S' \subseteq M$. We're interested in $S \cap S'$.
\end{enumerate}

Suppose $F \in Diff(M)$ (diffeomorphism on $M$). If we start with a submfd., can mapy with $S' = F(S)$ then we want to see how similar/different the submfd. is, like $S' \cap S$ and helps categorize $F$. 

Now, suppose $F_t \in Diff(M)$ and $t\in [0,1]$ and $F_{t=0} = Id$. Can keep track of how things move by looking at $S'_t = F_t(s)$ and looking at its intersection $S_t' \cap S$. 

Leads to \textbf{Fixed pt. Theory}, where $F(p)=p$. This theory can be expressed in terms of intersections of submfd. in $M \times M$  ($F: M \rightarrow M$).  Here, we have
\begin{itemize}
    \item Diagonal $\Delta = \{ (p,p) \in M\times M | p \in M\}$
    \item $\Gamma_F = \{ (p, F(p))| p \in M\}$.
\end{itemize}

Basically, looking for $\Delta \cap \Gamma_F$. Number of fixed points corresponds to diagonal between.

\textbf{Question:} Suppose we have two embedded submfds $S, S' \subseteq M$, what can we say about theiri intersection $S\cap S'$? 

Ex. Living in $M= \mathbb{R}^2$. If $S$ is a sine wave adn $S'$ is cosine, then intersections happen to be points. 

If one curve is on the axis and another touches, lingers, and leaves, then intersection will be points and an interval. $S \cap S' = \{$ pts. and intervals$\}$. This is not a mfd. since the intersection space is not a single dimension!

Can ask question: Is there a condition that can guarantee that the intersection is also a submfd.? 

Answer: yes, by the transversal intersection. 

\textbf{Defn:} Let $M$ be a smooth mfd. We say two embedded submfds, $S, S' \subseteq M$ intersect transverseally if for each $p \in S \cap S'$, the two tangent spaces $T_P M \subseteq T_P M$ and $T_P S' \subseteq T_P M$ together span $T_P M$. 

Often it is useful to consider one of the embedded mfd. as a smooth embedding. \textbf{Defn.:} Lett $F: N \rightarrow M$ be a smooth map and $S \subseteq M$ be an embedded submfd. Then, we say $F$ is transverse to $S$ if for every $p \in F^{-1}(S)$, the tangent spaces $dF_P(T_P N)$ and $T_{F(P)} S$ together span $T_{F(P)} M$. (see image from class).

\textbf{Remarks}
\begin{itemize}
    \item Since embedded submfd. $F(N) = S'$ can be considered a smooth embedding by the inclusion map of $i: S' \rightarrow M$ we  have two embedded submfds. intersect transversely. This is equivalent to the inclusion of either submfd. is transverse to the other. 
    \item If $F$ is a smooth submersion, (i.e. $dF_P(TN)$ spans $T_{F(P)}M$) then $F$ is trivially transverse to every embedded submfd. of $M$. 
\end{itemize}

\textbf{Thrm.:} Let $N$ and $M$ be smooth mfds. and $S \subseteq M$ is an embedded submfd. 
\begin{enumerate}
    \item If $F: N \rightarrow M$ is a smooth amp that is transverse to $S$, then, $F^{-1}(S)$ is an embedded submfd. of $N$ and, moreover, codimension (in $N$), $F^{-1}(S) = codim S$, (In $M$) - this tells the dimensions since, if you know the codimension, you know the dimensionality.
    \item If $S' \subseteq M$ is an embedded submfd. that intersects $S$ transversely, then $S \cap S'$ is an embedded submfd. of $M$, where codimension of $S \cap S' = codim S + codim S' $.
\end{enumerate}
We need to check the dimension. \textbf{Remark:} 
\begin{itemize}
    \item Dimension of $S \cap S'$. Let $m = \dim M$ and $n = \dim N$, then $s =\dim S$ and $S' = \dim S'$ and $k = \dim S \cap S'$. 
    \item Then $codim S \cap S' = codim S+ codim S'$. Thinking of dimensions, this is equivalent to $m -k = (m-s) + (m - S')$, which gives us $S + S' = m+k$. We're in 3D with two planes, $S$ and $S'$, can check if in 3D, by $2 + 2 = 3 + 1$, where $k=1$. 
\end{itemize}

We'll prove the thrm. next time.

\end{document}
